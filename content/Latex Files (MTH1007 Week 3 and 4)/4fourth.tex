\documentclass{article}

\usepackage{amsmath}  %additional packages for better styles
\usepackage{amsthm}

\newtheorem{thrm}{Theorem} %theorem-type environments declarations
\newtheorem{lemma}{Lemma}
\newtheorem{crly}{Corollary}


\begin{document}
\title{Environments in LATEX}
\date{}
\maketitle


\textbf{Environments.} One of the important features of LATEX.
%Begin with \begin{...} and end with \end{...}.
There are standard environments, like theorem-type, or  equation.
Usually generates a number.
%For cross-references include \label{..} which you choose to be
%unique for each equation,  theorem, etc.

\textbf{Equation:} Example:
\begin{equation} \label{lbl}
a=b .
\end{equation}
Then we can refer to it by ``In equation (\ref{lbl})...''. But much smarter using \eqref{lbl}, which only works  with additional package amsmath in top-matter ---  it is  better since it makes (..)  and makes it always upright.

\textbf{Theorem-type:} These environments must be declared in `top-matter' --- see at the top of the input page. For example, first brace thrm is the name (you choose) of this theorem-type environment and second Theorem tells LATEX what you choose to type as a title, followed by number.

Theorem environment makes vertical space around statement, italicizes the statement, gives title like Theorem 1, Lemma 2, etc. Example:

\begin{thrm}\label{th1}
If $f$ is a function whose derivative exists at every point, then $f$
is a continuous function.
\end{thrm}

\begin{lemma}\label{l1}
If $f$ is a function whose derivative exists at a point $x_0$, then $f$
is continuous at this point.
\end{lemma}

We saw in Lemma~\ref{l1} that... .

\

\textbf{Proof:} if using package amsthm (as we do here, see at the top), there is a special environment `proof':
 produces nice spacing and qed symbol at the end:

\begin{proof}
Since ... we have ...

Then............ whatever. The result follows.
\end{proof}


\end{document}

