\documentclass{article}
%\usepackage{amsmath}
%\usepackage{amsthm}

\begin{document}
\title{Tables in LATEX}
\date{}
\maketitle

 %Environment \begin{tabular}{<parameters>}.. \end{tabular}
 %produces simple tables. Here `parameters' is a sequence of
 %symbols  l,   r,   c  in any order (meaning  left-, right-justified,
 %centred columns). Optionally vertical line | between these l, r, c
 %makes vertical divider. In the input file, in a given row,
 %column elements are separated by the special symbol \&, and a
 %new row is started by \\
% Horizontal lines are produced by \hline (between lines).
%Note that the special symbol & 
%must appear the same number of times for each row!!!!!!!!!!!!}
Table with three columns, vertical line between first and second columns, and text is right-justified in 1st column, centred in 2nd, left-justified in third:



\begin{tabular}{|r|cl|}
\hline
Maths& 23& students\\
Physics&12&PhDs\\
Comp. Science&123 staff& mixed\\
\hline\hline
Others&& \\
\hline
\end{tabular}



Tabular environment can be used without lines to simply arrange text in justified columns, like

\

\begin{tabular}{l c l c l}

BSc Maths& &Bsc Maths and Physics && Bsc Physics\\
BSc Maths and Comp. Sci. & &MMath &&MMath Maths and Physics\\
MPhysics &&PhD in Maths&& PhD in Physics
\end{tabular}

\

Here extra empty columns were added to make more space between columns.


%Instead of left-justified option l, one can use p{<width>}, which will
Wrap left-justified text with given width, useful if text is long. Example:



\begin{tabular}{|l|rp{6cm}|}
\hline
Maths& 23& students who are all very clever and eager to learn maths\\
\hline
Physics&12&PhD students who are all very clever and eager to do research \\
\hline
Comp. Science&123 & members of staff who are all highly qualified computer scientists
\\
\hline
\end{tabular}

\


There are many other tricks for tables, how to make rows thicker, how to span rows or columns, etc., see manuals if interested.


\textbf{Table as a `float'.}


To have a table nicely (automatically) embedded in the text, and have number, caption,
use the following table environment (where square  brackets contain optional position `wish' (t for `top', b for `bottom', h for `here'):

\begin{table}[h]\caption{Listing}\label{t1} \centering %ambient  table environment
\begin{tabular}{|r|cl|} %tabular environment
\hline
Maths& 23& students\\
Physics&12&PhDs\\
\hline\hline
Others&& \\
\hline
\end{tabular}
\end{table}

Note that label is placed after caption.

Then we refer like ``In Table~\ref{t1} we have ...''


\end{document}
