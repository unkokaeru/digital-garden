% This is a small sample LaTeX input file
%
% Use this file as a model for making your own LaTeX input file.
% Everything to the right of a  %  is a comment to you and is ignored by LaTeX.
%
% WARNING!  Do not type any of the following 10 characters except as directed:
%                &   $   #   %   _   {   }   ^   ~   \

\documentclass{article}        % Your input file must contain these two lines
\begin{document}               % plus the \end{document} command at the end.


\section{Simple Text}          % This command makes a section title with number.


\section*{Simple Text}          % ... without number.

Words are separated by one     or more           spaces.  Paragraphs are separated by
one or more blank lines.  The output
is not affected by           adding extra
spaces  in a line or       extra blank          lines to the input file.


Double quotes are typed like this: ``quoted text''.
Single quotes are typed like this: `single-quoted text'.

Long dashes are typed as three dash characters---like this.

Emphasized text is typed like this: \emph{this is emphasized}.

Bold       text is typed like this: \textbf{this is bold}.

Usually size of letters (font) is determined by which part of text it is, `section' or `title', etc. But you can force larger or smaller letters by using commands like {\large AAA}, {\Large AAA}, {\LARGE AAA}, {\Huge AAAA}. Or {\small AAAAA}, {\tiny AAAA}.


\subsection{A Warning or Two}  % This command makes a subsection title.

Note that every left brace in commands (specifying the scope) like in {\Large MMM}, must be matched with right brace. Otherwise an error, or unintended result.

\subsubsection{Memento}

Remember, don't type the 10 special characters (such as dollar sign and
backslash) except as parts of commands where they belong!  The following seven are printed by
typing a backslash in front of them:  \$  \&  \#  \%  \_  \{  and  \}.
The manual tells how to make other symbols, like \textbackslash

\end{document}                 % The input file ends with this command.
