% A simple article to illustrate document structure.
\documentclass{article}

\begin{document}

% ARTICLE TITLE, AUTHOR, ABSTRACT 

\title{How to Structure a LATEX Document}

\author{Andrew Roberts\\ %doublebackslash forces new line
        School of Computing,\\
		University of Leeds,\\
		Leeds,\\
		United Kingdom,\\
		LS2 1HE\\
		\texttt{andyr@comp.leeds.ac.uk}}  %\texttt formats the text to a typewriter style font

\date{32/11/16} %joke:-), can use \date{\today} \today is replaced with the current date

\maketitle %necessary command to make title

\begin{abstract}
In this article, I shall discuss some of the fundamental topics in
producing a structured document.  This document itself does not go into
much depth, but is instead the output of an example of how to implement
structure.
\end{abstract}

\section{\label{intro} Introduction} %\label{..} are optional "markers" for cross-referencing

This small document is designed to illustrate how easy it is to create a
well structured document within LATEX, see \cite{lamport94}.  You should quickly be able to
see how the article looks very professional, despite the content being
far from academic.  Titles, section headings, justified text, text
formatting etc., is all there, and you would be surprised when you see
just how little markup was required to get this output.

\section{\label{str} Structure}
One of the great advantages of LATEX is that all it needs to know is
the structure of a document, and then it will take care of the layout
and presentation itself.  So, here we shall begin looking at how exactly
you tell LATEX what it needs to know about your document.

\subsection{\label{top} Top Matter}
The first thing you normally have is a title of the document, as well as
information about the author and date of publication.  In LATEX terms,
this is all generally referred to as \emph{top matter}.

\subsubsection{\label{ai} Article Information}

 See the commands at the beginning for title, etc.

\subsubsection{\label{auth} Author Information}

It is common to not only include the author name, but to insert new
lines after and add things such
as address and email details.



\subsection{\label{sect} Sectioning Commands}
The commands for inserting sections are fairly intuitive.  Of course,
certain commands are appropriate to different document classes.
For example, a book has chapters but a article doesn't.

Numbering of the sections is performed automatically by LATEX, so don't
bother adding them explicitly, just insert the heading you want between
the curly braces.

If you don't want sections number, then add an asterisk (*) after the
(sub)section command, but before the first curly brace, e.g.,

\subsection*{A Title Without Numbers}


\subsection{\label{lbls} Labels} Use labels if you want cross-referencing. Like ``We discussed author info in Subsection~\ref{auth}.''  This is a really useful feature of LATEX, everything is  taken care of, even if you change structure, add more (sub)sections, etc. The same applies to bibliography citing, which has different format \cite{lamport94}, see below, --- even if you later add more bibliography, all references will be correct.

NOTE: for labels to become known to LATEX, building procedure should be done \textbf{twice}.

%Create the environment for the bibliography.  Since there are only two 
%references, set the label width to be one character (I shall follow
%convention as use the number '9'.  This is because it helps to remind
%that it is the maximum number of refs that is now permitted by that
%width).
\begin{thebibliography}{9} %number in brackets indicates maximum thickness
% of number; so if there are less than 10, put 9, if less than 100, put 99k

%The \bibitem is to start a new reference.
%Ensure that the cite_key=label in braces after \bibitem,  is
%unique.  Then refer in the text like  \cite{lamport94}, see above.

\bibitem{lamport94}
	  Leslie Lamport,  \emph{LATEX: A Document Preparation System}. 	  Addison Wesley, Massachusetts, 	  2nd Edition, 	  1994.

\bibitem{b2}
A. U. Thor, How to write, \emph{J. Good Writing}, \textbf{12}, no. 3 (2001), 12--45.

\end{thebibliography} %Must end the environment

\end{document}  %End of document.
